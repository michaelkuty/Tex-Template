%% LyX 1.5.5 created this file.  For more info, see http://www.lyx.org/.
%% Do not edit unless you really know what you are doing.
\documentclass[a4paper,czech,czech,openright,cleardoubleempty,BCOR10mm,DIV11]{scrreprt}
\usepackage[T1]{fontenc}
\usepackage[utf8]{inputenc}
\usepackage{array}
\usepackage{longtable}
\usepackage{varioref}
\usepackage{wrapfig}
\usepackage{fancybox}
\usepackage{calc}
\usepackage{framed}
\usepackage{url}
\usepackage{graphicx}
\usepackage{placeins} %floatbarrier \FloatBarrier
%\usepackage{listing}
\usepackage{pdfpages}

\makeatletter

\usepackage[font=small,labelfont=bf]{caption} %captiony

%%%%%%%%%%%%%%%%%%%%%%%%%%%%%% LyX specific LaTeX coěmmands.
\providecommand{\LyX}{L\kern-.1667em\lower.25em\hbox{Y}\kern-.125emX\@}
\newcommand{\lyxline}[1][1pt]{%
  \par\noindent%
  \rule[.5ex]{\linewidth}{#1}\par}
\newcommand{\noun}[1]{\textsc{#1}}
%% Special footnote code from the package 'stblftnt.sty'
%% Author: Robin Fairbairns -- Last revised Dec 13 1996
\let\SF@@footnote\footnote
\def\footnote{\ifx\protect\@typeset@protect
    \expandafter\SF@@footnote
  \else
    \expandafter\SF@gobble@opt
  \fi
}

\renewcommand{\baselinestretch}{1.2} %radkovani s

\expandafter\def\csname SF@gobble@opt \endcsname{\@ifnextchar[%]
  \SF@gobble@twobracket
  \@gobble
}
\edef\SF@gobble@opt{\noexpand\protect
  \expandafter\noexpand\csname SF@gobble@opt \endcsname}
\def\SF@gobble@twobracket[#1]#2{}
%% Because html converters don't know tabularnewline
\providecommand{\tabularnewline}{\\}

%%%%%%%%%%%%%%%%%%%%%%%%%%%%%% Textclass specific LaTeX commands.
\newenvironment{lyxcode}
{\begin{list}{}{
\setlength{\rightmargin}{\leftmargin}
\setlength{\listparindent}{0pt}% needed for AMS classes
\raggedright
\setlength{\itemsep}{0pt}
\setlength{\parsep}{0pt}
\normalfont\ttfamily}%
 \item[]}
{\end{list}}

%%%%%%%%%%%%%%%%%%%%%%%%%%%%%% User specified LaTeX commands.
%<-------------------------------společná nastavení------------------------------>
\usepackage[english]{babel}%počeštění názvů (Obsah, Kapitola, Literatura atp.)
\usepackage[]{hyperref} %odkazy v  pdf jsou klikací s barevnými rámečky
\usepackage[numbers,sort&compress]{natbib} %balíček pro citace literatury  
\usepackage{hypernat}%interakce mezi hyperref a natbib
\newcommand{\BibTeX}{{\sc Bib}\TeX}%BibTeX logo
\hypersetup{   % Nastavení polí PDF dokumentu 
pdftitle={Robotice},%   
pdfauthor={Michael Kutý & Aleš Komárek },%  
pdfsubject={},%   
pdfkeywords={}%                             
}
\usepackage{multicol}




%<-----------------------------volání stylů----------------------------------------->
% (znak % je označení komentáře: co je za ním, není aktivní)
%<------------------------------------písmo----------------------------------------->
%\usepackage{packages/bc-latinmodern}
%\usepackage{packages/bc-times}
\usepackage{packages/bc-palatino}
%\usepackage{packages/bc-iwona}
%\usepackage{packages/bc-helvetika}


%<------------------------------záhlaví stránek------------------------------------>
%\usepackage{packages/bc-headings}
\usepackage{packages/bc-fancyhdr}

%<------------------------------hlavičky kapitol------------------------------------>
%\usepackage{packages/bc-neueskapitel}
%\usepackage{packages/bc-fancychap}

\makeatother

\usepackage{babel}

%java code block%

\usepackage{listing}
\usepackage{listings}
\usepackage{color}

\definecolor{dkgreen}{rgb}{0,0.6,0}
\definecolor{gray}{rgb}{0.5,0.5,0.5}
\definecolor{mauve}{rgb}{0.58,0,0.82}

%\renewcommand*{\lstlistingname}{Ukázka kódu} %prejmenovani 
%\renewcommand*{\lstlistlistingname}{Seznam ukázek kódu}

% syntax highlight pro jazyk Java %
\lstset{
  %frame=r,
  captionpos=b,
  language=Java,
  aboveskip=3mm,
  belowskip=3mm,
  xleftmargin=0.2mm,
  showstringspaces=false,
  columns=flexible,
  basicstyle={\small\ttfamily},
  numbers=none,
  numberstyle=\tiny\color{gray},
  keywordstyle=\color{blue},
  commentstyle=\color{dkgreen},
  stringstyle=\color{mauve},
  breaklines=true,
  breakatwhitespace=true,
  tabsize=3,
    inputencoding=utf8,
    extendedchars=true,
    literate=%
    {á}{{\'a}}1
    {č}{{\v{c}}}1
    {ď}{{\v{d}}}1
    {é}{{\'e}}1
    {ě}{{\v{e}}}1
    {í}{{\'i}}1
    {ň}{{\v{n}}}1
    {ó}{{\'o}}1
    {ř}{{\v{r}}}1
    {š}{{\v{s}}}1
    {ť}{{\v{t}}}1
    {ú}{{\'u}}1
    {ů}{{\r{u}}}1
    {ý}{{\'y}}1
    {ž}{{\v{z}}}1
    {Á}{{\'A}}1
    {Č}{{\v{C}}}1
    {Ď}{{\v{D}}}1
    {É}{{\'E}}1
    {Ě}{{\v{E}}}1
    {Í}{{\'I}}1
    {Ň}{{\v{N}}}1
    {Ó}{{\'O}}1
    {Ř}{{\v{R}}}1
    {Š}{{\v{S}}}1
    {Ť}{{\v{T}}}1
    {Ú}{{\'U}}1
    {Ů}{{\r{U}}}1
    {Ý}{{\'Y}}1
    {Ž}{{\v{Z}}}1
}

\begin{document}
%~\thispagestyle{empty}{\small ~\vfill{}
%}{\small \par}

%~\thispagestyle{empty}\vfill{}
%Tato stránka je tzv. protititul a je graficky součástí titulní stránky.
%Nechte ji prázdnou, nebo na ni umístěte vhodnou fotografii či ilustraci.

\cleardoublepage{}~\thispagestyle{empty}\begin{center}\pagenumbering{roman}\vspace{10mm}


\textsf{\textsc{\noun{\LARGE University of Hradec Kralove}}}\\
\vspace{0.5em}
\textsc{\noun{\LARGE Faculty of Informatics and Management}}\\
\vspace*{1em}
\textsf{\textsc{\noun{\Large Department of Information Technologies }}}

\vspace{15mm}

%\includegraphics[width=0.4\textwidth]{logo/uhk}

\vspace{15mm}


\textsf{\huge Project Robotice}{\huge \par}

\vspace{15mm}


\textsf{\LARGE }{\LARGE \par}

\vspace{10mm}


\end{center} 

\vspace*{\fill}


\vspace{10mm}


\begin{description}
\item [{{\large Authors:}}] \noindent \textsf{\large Michael Kuty and Ales Komarek}{\large \par}

{\large \par}
\end{description}


%{\small \thispagestyle{plain}\addcontentsline{toc}{chapter}{Abstrakt} }{\small \par}

\newpage{}\thispagestyle{plain}

{\small %\setcounter{page}{3} % nastavení číslování stránek
\ }{\small \par}

\noindent {\small \vfill{}
 % nastavuje dynamické umístění následujícího textu do spodní části stránky
~}{\small \par}

%\newpage{}\thispagestyle{plain}

{\small %\setcounter{page}{3} % nastavení číslování stránek
\ }{\small \par}

\noindent {\small \vfill{}
 % nastavuje dynamické umístění následujícího textu do spodní části stránky
~}{\small \par}

\subsubsection{Abstract}

Robotice is a framework for robotics, physical computing, and the Internet of Things, written in the Python programming language.

\cleardoublepage{}

%\thispagestyle{empty}~{\small \addcontentsline{toc}{chapter}{Zadání
%práce} }{\small \par}

{\small %%%   Výtisk pak na tomto míste nezapomeňte PODEPSAT!
%%%                                         *********
}{\small \par}

\cleardoublepage{}\thispagestyle{empty}{\small 
%\setcounter{secnumdepth}{3}
%\setcounter{tocdepth}{2}%hloubla obsahu
\tableofcontents{}% vkládá automaticky generovaný obsah dokumentu
\cleardoublepage{}}{\small \par}

\pagenumbering{arabic}%start arabic pagination from 1 


\chapter{Introduction}

...

\section{Motivation}

...

\section{Problem Statement}

...


\section{Goals}

...

\section{Related Projects}

...

\section{Structure of Project}

...



\chapter{Robotice}

\section{Introduction}

Python daemon for collecting data from peripherals(sensors, ..) and sending instructions into external actuators.

\subsection{Motivation}

Lowermost layer of Robotice which must resolve differences in single-board computers(BB, Rpi, etc.). We must have driver for every device and every platform.

\subsection{Goals}

\begin{description}

\item[Collecting and Acting] - GPIO, I2C drivers etc.
\item[Autonomous] - collecting and acting must work without internet connection
\item[Support] - support for most of OS and SB computers

\end{description}

\subsection{Problem Statement}

\begin{itemize}

\item{Drivers for external peripheral devices(GPIO, I2C, ..)}
\item{Differences in single-board computers}
\item{Differences on the OS Layer(pkgs, etc.)}

\end{itemize}

\subsection{Related Projects}

\subsubsection{Cylon.JS}

Platform builded on the Node.js under JavaScript language. Supported is about thirty platforms, as BeagleBone, Rpi, and Drone.

\subsubsection{Artoo}

Ruby copy of Cylon.JS

\subsubsection{Gobot}

Gobot is a framework for robotics, physical computing, and the Internet of Things, written in the Go programming language

\subsection{Structure of Project}

...


\chapter{Robotice Control}


\section{Introduction}

Django based application for easily controlling every system powered by robotice.

\subsection{Motivation}

Independent application for controling robotice systems over AMPQ.

\subsection{Goals}

\begin{description}

\item[Web interface] - Adding new device etc.
\item[REST API] - API for all robotice features

\end{description}

\subsection{Problem Statement}

---

\subsection{Related Projects}


\subsection{Structure of Project}

...

\include{dashboard}

\include{brain}

\begin{thebibliography}{10}

%\bibitem{whoControlVertx}Phipps, Simon \emph{Who controls Vert.x: Red Hat, VMware, or neither?}
%{[}online]. {[}cit. 2014-02-16]. Dostupný z WWW: \url{http://www.infoworld.com/d/open-source-software/who-controls-vertx-red-hat-vmware-or-neither-210549}


\end{thebibliography}

\addcontentsline{toc}{chapter}{Bibliography} 

\cleardoublepage{}

\appendix
\pagenumbering{Roman}

\part*{Attachments}

\listoffigures

\listoftables

\lstlistoflistings



\end{document}
